\documentclass[a4paper]{article}

%\VignetteIndexEntry{Introduction to ggcluster}
%\VignettePackage{ggcluster}

% Definitions
%\newcommand{\slan}{{\tt S}}
%\newcommand{\rlan}{{\tt R}}
\newcommand{\ggcluster}{{\tt ggcluster}}
\newcommand{\code}[1]{{\tt #1}}
\setlength{\parindent}{0in}
\setlength{\parskip}{.1in}
\setlength{\textwidth}{140mm}
\setlength{\oddsidemargin}{10mm}

\title{Introduction to the \ggcluster{} package for plotting cluster analysis results}
\author{Andrie de Vries}

\usepackage{Sweave}
\begin{document}

\maketitle

\ggcluster{} is a package that makes it easy to plot extract results of cluster analysis into a data frame.  It provides a common framework for a number of hierarchical and other cluster analysis techniques, including:
\itemize{
\item hclust() - hierarchical clustering
\item dendrogram()- dendrograms
\item kmeans() - k-means clustering
\item Mclust() - model based clustering
}

\section{Introduction}

Most of the cluster analysis functions in R, e.g. \code{hclust()} and \code{dendrogram()} provide \code{plot()} functions to display the results.  However, it is sometimes hard to extract the data from this analysis to customise these plots, especially when the \code{plot()} functions don't provide an option to return the data.

The \ggluster{} package provides a general framework to extract the plot data for a variety of cluster analysis functions.

It does this by providing generic function \code{cluster_data()} that will extract the appropriate cluster data as well as labels.  This data is returned as a list of data.frames.  These data frames can 
  


\section{Dendrograms}

The \code{hclust()} and \code{dendrogram()} functions in R makes it easy to plot the results of hierarchical cluster analysis and other dendrograms in R.  However, it is hard to extract the data from this analysis to customise these plots, since the \code{plot()} functions for both these classes prints directly without the option of returning the plot data.  

\begin{Schunk}
\begin{Sinput}
> library(ggplot2)
> library(ggcluster)
\end{Sinput}
\end{Schunk}

To plot a dendrogram:

\begin{Schunk}
\begin{Sinput}
> hc <- hclust(dist(USArrests), "ave")
> dhc <- as.dendrogram(hc)
> ddata <- cluster_data(dhc, type = "rectangle")
> p <- ggplot(ddata$segments) + geom_segment(aes(x = x0, y = y0, 
+     xend = x1, yend = y1)) + coord_flip() + scale_y_reverse(expand = c(0.2, 
+     0))
> print(p)
\end{Sinput}
\end{Schunk}

\begin{figure}[p]
\begin{center}
{
\includegraphics[width=4in, height=3in]{ggcluster-dendro}
}
\end{center}
\caption{A dendrogram produced using \code{cluster_data()} and \code{ggplot()}}
\end{figure}


\section{Cluster analysis}

Some text here.

\section{Final thoughts}

The last word.


% Start a new page
% Not echoed, not evaluated
% ONLY here for checkVignettes so that all output doesn't
% end up on one enormous page

\end{document}




